\documentclass[12pt,letterpaper]{article}
\usepackage{fullpage}
\usepackage[top=2cm, bottom=4.5cm, left=2.5cm, right=2.5cm]{geometry}
\usepackage{amsmath,amsthm,amsfonts,amssymb,amscd}
% \usepackage{lastpage}
\usepackage{enumerate}
\usepackage{fancyhdr}
% \usepackage{mathrsfs}
\usepackage{xcolor}
\usepackage{graphicx}
\usepackage{listings}
\usepackage{hyperref}

\usepackage{float}

% define vector
\newcommand{\q}{\underline}
\newcommand{\mt}{\mathrm}

\setlength{\parindent}{0.2in}
\setlength{\parskip}{0.1in}

% Edit these as appropriate
\newcommand\course{Phys 213}
\newcommand\hwnumber{1}                  % <-- homework number
\newcommand\NetIDa{M.-F. Ho}           % <-- NetID of person #1
% \newcommand\NetIDb{netid12038}           % <-- NetID of person #2 (Comment this line out for problem sets)

\pagestyle{fancyplain}
\headheight 35pt
\lhead{\NetIDa}
% \lhead{\NetIDa\\\NetIDb}                 % <-- Comment this line out for problem sets (make sure you are person #1)
\chead{\textbf{\Large Homework 6}}
\rhead{\course \\ \today}
\lfoot{}
\cfoot{}
\rfoot{\small\thepage}
\headsep 1.5em

\newcommand{\Data}{\mathcal{D}}
\newcommand{\xvec}{\boldsymbol{x}}
\newcommand{\Xvec}{\boldsymbol{X}}
\newcommand{\Var}{\textrm{Var}}
\newcommand{\normal}{\textrm{N}}
\newcommand{\xmean}{\langle \xvec \rangle}
\newcommand{\newx}{\tilde{x}}
\newcommand{\integer}{\mathbb{N}}
\newcommand{\thetarv}{\tilde{\theta}}
\newcommand{\phirv}{\tilde{\phi}}

\newcommand{\ml}{m_{\ell}}
\newcommand{\specterms}{^{2S+1}\mathcal{L}^p_\mathcal{J}}

\newcommand{\hi}{\textrm{H\,I}}
\newcommand{\cm}{\textrm{\,cm}}
\newcommand{\cms}{\textrm{\,cm/s}}
\newcommand{\kms}{\textrm{\,km/s}}
\newcommand{\cmcm}{\textrm{\,cm}^{-2}}
\newcommand{\hz}{\textrm{\,s}^{-1}}

\newcommand{\civ}{\textrm{C\,IV}}
\newcommand{\mgii}{\textrm{Mg\,II}}
\newcommand{\caii}{\textrm{Ca\,II}}
\newcommand{\siii}{\textrm{Si\,II}}

\newcommand{\Isky}{I_\nu^{\textrm{sky}}}
\newcommand{\Iradio}{I_\nu^{\textrm{radio}}}
\newcommand{\indicator}{\mathbb{I}}
\newcommand{\Snuradio}{S_\nu^{\textrm{radio}}}

\begin{document}

\section*{1: 21 cm {\hi} gas}
Show:
\begin{equation}
    M_{\hi} = \frac{16 \pi m_{H}}{3 A_{u \ell} h \nu_{u \ell}} D_{L}^2 F_{obs}
\end{equation}

If we decompose the equation into:
\begin{equation*}
    \begin{split}
        M_{\hi} &=
        \left( \frac{16 \pi}{3 A_{u \ell} h \nu_{u \ell}} \right)
        m_{H} 
        (D_{L}^2 F_{obs}) \\
        &= 
        \left( \frac{16 \pi}{3 A_{u \ell} h \nu_{u \ell}} 
        \int [I_\nu - I_\nu(0)]d\nu
        \right)
        m_{H} 
        \times
        (\textrm{area})\\
        &= 
        \left( \textrm{column density}\right) (\textrm{mass per particle})(\textrm{area}).
    \end{split}
\end{equation*}

It follows the line of thought from  (8.16):
\begin{equation}
    \begin{split}
        \int [I_\nu - I_\nu(0)] d\nu &= 
        \frac{3}{16 \pi} A_{u \ell} h \nu_{u \ell} N_{\hi}\\
        \Rightarrow 
        N_{\hi} &= \frac{16 \pi}{3 A_{u\ell} h \nu_{u\ell}} 
        \int [I_{\nu} - I_{\nu}(0)] d\nu
    \end{split}
\end{equation}
where we integrate the intensity over line to get the expression of column density.

Some unit conversions could help to think this problem:
\begin{equation*}
    \begin{split}
        I_\nu &= L_{\nu} / (\textrm{area (of the object)})\\
        L_{\nu} &\propto D_{L}^2 F_{\nu},
    \end{split}
\end{equation*}
thus
\begin{equation}
    D_{L}^2 F_{obs} = D_{L}^2 \int F_{\nu} d\nu
    = D_{L}^2 \int \frac{L_{\nu}}{D_{L}^2} d\nu
    = \int I_\nu d\nu \times (\textrm{area}).
\end{equation}

Reverse the logic:
\begin{equation}
    \begin{split}
        M_{\hi} &= N_{\hi} \times m_{H} \times (\textrm{area})\\
        &= \left( \frac{16 \pi}{3 A_{u \ell} h \nu_{u \ell}} 
        \int [I_\nu - I_\nu(0)]d\nu
        \right)
        m_{H} 
        \times
        (\textrm{area}) \\
        &= \left( \frac{16 \pi}{3 A_{u \ell} h \nu_{u \ell}} \right)
        m_{H} 
        (D_{L}^2 F_{obs})
    \end{split}
\end{equation}

\section*{8.4: phases of $\hi$}

\subsection*{(a): radiative transfer}
The radiative transfer equation is:
\begin{equation}
    dI_\nu = -I_\nu \kappa_\nu ds + j_\nu ds.
\end{equation}
Consider the case for 21 centimeter signals, we have:
\begin{equation*}
    \begin{split}
        j_\nu &\simeq \frac{3}{16 \pi} A_{u \ell} h \nu_{u \ell} n(\hi) \phi_\nu\\
        \kappa_\nu &\simeq \frac{3}{32 \pi} A_{u \ell} \frac{h c \lambda_u \ell}{k T_{\textrm{spin}}} n(\hi) \phi_\nu.
    \end{split}
\end{equation*}

The integral form of the radiative transfer equation is:
\begin{equation}
    I_\nu (\tau_\nu) = I_\nu (0) e^{-\tau_\nu} 
    + \int_0^{\tau_\nu} e^{-(\tau_\nu - \tau')} S_\nu d\tau'.
\end{equation}
The interpretation of this equation is the intensity at $\tau_\nu$ is equal to initial intensity attenuated by a exponential factor $e^{-\tau_\nu}$ and the emission of the gas in between integral over the optical depth in between and attenuated by a exponential factor $e^{-(\tau_\nu - \tau')}$.
The source function $S_\nu = j_\nu / \kappa_\nu$ depends on the material in between.

We could imagine $I_\nu(0) = \Isky + \Iradio$.
We could also imagine the integral should be decomposed into cold gas area and warm gas area.

For case one, the factorization should be:
\begin{equation*}
    \int_0^{\tau_\nu} = \int_0^{\tau_w} + \int_{\tau_w}^{\tau_c + \tau_w}.
\end{equation*}
For case two, the factorization is:
\begin{equation*}
    \int_0^{\tau_\nu} = \int_0^{\tau_c} + \int_{\tau_c}^{\tau_c + \tau_w}.
\end{equation*}

The source function for 21 centimeter signal is:
\begin{equation}
    \begin{split}
        S_\nu(\tilde{T}_{spin}, \nu_{u \ell}, \lambda_{u \ell}) = \frac{j_\nu}{\kappa_\nu}
        \simeq \frac{2 k T_{spin} \nu_{u \ell}}{c \lambda_{u \ell}}.            
    \end{split}
\end{equation}

Combine all information, we have:
\begin{equation}
    \begin{split}
        I_{\nu}(\tilde{\tau_\nu} &= \tau_w + \tau_c)\\
        &= (\Isky + \Iradio)e^{-(\tau_c + \tau_w)}\\
        &+ 
        \int_0^{\tau_1} e^{-(\tau_c + \tau_w) - \tau'} 
        \frac{2 k T_{1} \nu_{u \ell}}{c \lambda_{u \ell}} d\tau'
        + 
        \int_{\tau_1}^{\tau_w + \tau_c} e^{-(\tau_c + \tau_w) - \tau'} 
        \frac{2 k T_{2} \nu_{u \ell}}{c \lambda_{u \ell}} d\tau',
    \end{split}
\end{equation}
where
\begin{equation}
    \begin{split}
        \tau_1 &= \tau_w \indicator(\textrm{case1}) + \tau_c \indicator(\textrm{case2})\\
        T_1 &= T_w \indicator(\textrm{case1}) + T_c \indicator(\textrm{case2})\\
        T_2 &= T_w \indicator(\textrm{case2}) + T_c \indicator(\textrm{case1})\\,
    \end{split}
\end{equation}
where $\indicator$ is the indicator function.

Write the equation into flux density form, simply multiply by the beamsize $\Omega$:

\begin{equation}
    \begin{split}
        F_\nu^{*} &= \Omega I_{\nu}(\tilde{\tau_\nu} = \tau_w + \tau_c)\\
        &= \Omega (\Isky + \Iradio)e^{-(\tau_c + \tau_w)}\\
        &+ 
        \Omega\int_0^{\tau_1} e^{-(\tau_c + \tau_w) - \tau'} 
        \frac{2 k T_{1} \nu_{u \ell}}{c \lambda_{u \ell}} d\tau'
        + 
        \Omega\int_{\tau_1}^{\tau_w + \tau_c} e^{-(\tau_c + \tau_w) - \tau'} 
        \frac{2 k T_{2} \nu_{u \ell}}{c \lambda_{u \ell}} d\tau'.
    \end{split}
    \label{eq:flux_density}
\end{equation}

Note: the source function $S_\nu$ has the same notation as the $S_\nu$ (the flux density from the source in the absence of any intervening absorption) in the question.
So I denote $\Snuradio$ to be the $S_\nu$ given in the question.

\subsection*{(b): off}
If the telescope is pointing at the sky, there would be no photons from the source.
Re-write Eq~\ref{eq:flux_density} to fit this scenario:
\begin{equation}
    \begin{split}
        F_\nu^{\textrm{off}} &= \Omega I_{\nu}(\tilde{\tau_\nu} = \tau_w + \tau_c)\\
        &= \Omega \Isky e^{-(\tau_c + \tau_w)}\\
        &+ 
        \Omega\int_0^{\tau_1} e^{-(\tau_c + \tau_w) - \tau'} 
        \frac{2 k T_{1} \nu_{u \ell}}{c \lambda_{u \ell}} d\tau'
        + 
        \Omega\int_{\tau_1}^{\tau_w + \tau_c} e^{-(\tau_c + \tau_w) - \tau'} 
        \frac{2 k T_{2} \nu_{u \ell}}{c \lambda_{u \ell}} d\tau'.
    \end{split}
    \label{eq:off}    
\end{equation}

\subsection*{(c): $\Snuradio$ known}
Re-write Eq~\ref{eq:flux_density} to include $\Snuradio$:
\begin{equation}
    \begin{split}
        F_\nu^{*} &= \Omega I_{\nu}(\tilde{\tau_\nu} = \tau_w + \tau_c)\\
        &= (\Snuradio + \Omega\Isky)e^{-(\tau_c + \tau_w)}\\
        &+ 
        \Omega\int_0^{\tau_1} e^{-(\tau_c + \tau_w) - \tau'} 
        \frac{2 k T_{1} \nu_{u \ell}}{c \lambda_{u \ell}} d\tau'
        + 
        \Omega\int_{\tau_1}^{\tau_w + \tau_c} e^{-(\tau_c + \tau_w) - \tau'} 
        \frac{2 k T_{2} \nu_{u \ell}}{c \lambda_{u \ell}} d\tau'.
    \end{split}
\end{equation}

Now the flux of source and flux of sky are given:
\begin{equation}
    \begin{split}
        \tilde{F}^{\textrm{off}} &= F^{\textrm{off}}\\
        \tilde{F}^{\textrm{*}}   &= F^{\textrm{*}}\\
    \end{split}
\end{equation}

The obvious choice is to subtract between Eq~\ref{eq:flux_density} and Eq~\ref{eq:off}, so we get rid of the emission from the material in between.
\begin{equation}
    F_{\nu}^{*} - F^{\textrm{off}}_\nu
    = \Snuradio e^{-(\tilde{\tau}_c + \tilde{\tau}_w)}.
\end{equation}
A little bit algebra:
\begin{equation}
    \tilde{\tau}_c + \tilde{\tau}_w
    = - \log{( \frac{F_{\nu}^{*} - F^{\textrm{off}}_\nu}{\Snuradio} )}.
\end{equation}
Strange enough, we don't need the geometry of case 1 or case 2.

\subsection*{(d): column density}
The first way is to just integrate the equation given in the textbook:
\begin{equation*}
    \frac{dN(\hi)}{dv}
    \simeq \frac{32 \pi }{3 \lambda^2} \frac{k}{hc A_{u\ell}} [T^{\textrm{on}}(v) - T^{\textrm{sky}}(v)]
\end{equation*}
with the given linear relationship between antenna temperature and intensity. 
So the column density can be found by measurements is proved. 

The second way is to start with (8.14):
\begin{equation*}
    I_\nu = \Isky + \Iradio + \frac{3}{16 \pi} A_{u \ell} h \nu_{u\ell} \phi_\nu N_{\hi}.
\end{equation*}
Slight changes:
\begin{equation*}
    I_\nu^{\textrm{off}} - \Isky = \frac{3}{16 \pi} A_{u \ell} h \nu_{u\ell} \phi_\nu N_{\hi}.
\end{equation*}
The line profile is given:
\begin{equation*}
    \phi_\nu = \frac{1}{\sqrt{2\pi} \sigma_V} \frac{c}{\nu_{u \ell}} e^{-u^2 / 2 \sigma_V^2}.
\end{equation*}
For optical thin, the absorption is almost negligible, so the geometric factor is not parameterized in the equation.
Finally, the total column density could be obtained by integrate out the line profile:
\begin{equation}
    N_{\hi} = \frac{16 \pi}{3 A_{u\ell} h \nu_{u \ell}} \frac{\int I_\nu^{\textrm{off}} - \Isky d\nu}{ \int \phi_\nu d\nu },
\end{equation}
where $\Isky$, $u$ and $\ell$, and $\phi_\nu$ are known.
So the total column density could be found by measurements is proved.

\subsection*{(e) effective spin temperature}
Write the flux density equation again:
\begin{equation}
    \begin{split}
        F_\nu^{*} &= \Omega I_{\nu}(\tilde{\tau_\nu} = \tau_w + \tau_c)\\
        &= (\Snuradio + \Omega\Isky)e^{-(\tau_c + \tau_w)}\\
        &+ 
        \Omega\int_0^{\tau_1} e^{-(\tau_c + \tau_w) - \tau'} 
        \frac{2 k T_{1} \nu_{u \ell}}{c \lambda_{u \ell}} d\tau'
        + 
        \Omega\int_{\tau_1}^{\tau_w + \tau_c} e^{-(\tau_c + \tau_w) - \tau'} 
        \frac{2 k T_{2} \nu_{u \ell}}{c \lambda_{u \ell}} d\tau'.
    \end{split}
\end{equation}

Slightly re-write right hand side:
\begin{equation*}
    \begin{split}
        F_\nu^{*} &- (\Snuradio + \Omega\Isky)e^{-(\tau_c + \tau_w)}\\
        &= 
        T_{1}\Omega\int_0^{\tau_1} e^{-(\tau_c + \tau_w) - \tau'} 
        \frac{2 k \nu_{u \ell}}{c \lambda_{u \ell}} d\tau'
        + 
        T_{2} \Omega\int_{\tau_1}^{\tau_w + \tau_c} e^{-(\tau_c + \tau_w) - \tau'} 
        \frac{2 k  \nu_{u \ell}}{c \lambda_{u \ell}} d\tau'\\
        &=
        T_{\textrm{eff}}\Omega\int_0^{\tau_w + \tau_c} e^{-(\tau_c + \tau_w) - \tau'} 
        \frac{2 k \nu_{u \ell}}{c \lambda_{u \ell}} d\tau'
    \end{split}
\end{equation*}
The last line we assume an effective temperature which describes the combination of the behaviours of $T_1$ and $T_2$.
This implies:
\begin{equation}
    T_\textrm{eff}
    = \frac{T_{1}\int_0^{\tau_1} e^{-(\tau_c + \tau_w) - \tau'} 
    \frac{2 k \nu_{u \ell}}{c \lambda_{u \ell}} d\tau'
    +
    T_{2} \int_{\tau_1}^{\tau_w + \tau_c} e^{-(\tau_c + \tau_w) - \tau'} 
        \frac{2 k  \nu_{u \ell}}{c \lambda_{u \ell}} d\tau'}{
            \int_0^{\tau_w + \tau_c} e^{-(\tau_c + \tau_w) - \tau'} 
        \frac{2 k \nu_{u \ell}}{c \lambda_{u \ell}} d\tau'
        }.
\end{equation}
Note $\tau_w + \tau_c$ is known, $\ell$ and $u$ are known.
Also from (d):
\begin{equation}
    N_{\hi} = \frac{16 \pi}{3 A_{u\ell} h \nu_{u \ell}} \frac{\int I_\nu^{\textrm{off}} - \Isky d\nu}{ \int \phi_\nu d\nu },
    \label{eq:column}
\end{equation}
where
\begin{equation}
    I_\nu^{\textrm{off}}
    =
    \Isky e^{\tau_w + \tau_c} +
     T_{\textrm{eff}}\int_0^{\tau_w + \tau_c} e^{-(\tau_c + \tau_w) - \tau'} 
    \frac{2 k \nu_{u \ell}}{c \lambda_{u \ell}} d\tau'.
    \label{eq:effective_sky}
\end{equation}
Without making too complicate mathematics, from Eq~\ref{eq:column} and Eq\ref{eq:effective_sky} we can see the effective temperature also relates to the total column density.

\end{document}
