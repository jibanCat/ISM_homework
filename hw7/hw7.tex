\documentclass[12pt,letterpaper]{article}
\usepackage{fullpage}
\usepackage[top=2cm, bottom=4.5cm, left=2.5cm, right=2.5cm]{geometry}
\usepackage{amsmath,amsthm,amsfonts,amssymb,amscd}
% \usepackage{lastpage}
\usepackage{enumerate}
\usepackage{fancyhdr}
% \usepackage{mathrsfs}
\usepackage{xcolor}
\usepackage{graphicx}
\usepackage{listings}
\usepackage{hyperref}

\usepackage{float}

% define vector
\newcommand{\q}{\underline}
\newcommand{\mt}{\mathrm}

\setlength{\parindent}{0.2in}
\setlength{\parskip}{0.1in}

% Edit these as appropriate
\newcommand\course{Phys 213}
\newcommand\hwnumber{1}                  % <-- homework number
\newcommand\NetIDa{M.-F. Ho}           % <-- NetID of person #1
% \newcommand\NetIDb{netid12038}           % <-- NetID of person #2 (Comment this line out for problem sets)

\pagestyle{fancyplain}
\headheight 35pt
\lhead{\NetIDa}
% \lhead{\NetIDa\\\NetIDb}                 % <-- Comment this line out for problem sets (make sure you are person #1)
\chead{\textbf{\Large Homework 7}}
\rhead{\course \\ \today}
\lfoot{}
\cfoot{}
\rfoot{\small\thepage}
\headsep 1.5em

\newcommand{\Data}{\mathcal{D}}
\newcommand{\xvec}{\boldsymbol{x}}
\newcommand{\Xvec}{\boldsymbol{X}}
\newcommand{\Var}{\textrm{Var}}
\newcommand{\normal}{\textrm{N}}
\newcommand{\xmean}{\langle \xvec \rangle}
\newcommand{\newx}{\tilde{x}}
\newcommand{\integer}{\mathbb{N}}
\newcommand{\thetarv}{\tilde{\theta}}
\newcommand{\phirv}{\tilde{\phi}}

\newcommand{\ml}{m_{\ell}}
\newcommand{\specterms}{^{2S+1}\mathcal{L}^p_\mathcal{J}}

\newcommand{\hi}{\textrm{H\,I}}
\newcommand{\cm}{\textrm{\,cm}}
\newcommand{\cms}{\textrm{\,cm/s}}
\newcommand{\kms}{\textrm{\,km/s}}
\newcommand{\cmcm}{\textrm{\,cm}^{-2}}
% \newcommand{\hz}{\textrm{\,s}^{-1}}

\newcommand{\civ}{\textrm{C\,IV}}
\newcommand{\mgii}{\textrm{Mg\,II}}
\newcommand{\caii}{\textrm{Ca\,II}}
\newcommand{\siii}{\textrm{Si\,II}}

\newcommand{\Isky}{I_\nu^{\textrm{sky}}}
\newcommand{\Iradio}{I_\nu^{\textrm{radio}}}
\newcommand{\indicator}{\mathbb{I}}
\newcommand{\Snuradio}{S_\nu^{\textrm{radio}}}

\newcommand{\sradius}{R_{\textrm{S0}}}
\newcommand{\hdensity}{n(\textrm{H}^{+})}
\newcommand{\second}{\textrm{s}}

\newcommand{\Stroradius}{Str\"omgren radius}
\newcommand{\erg}{\textrm{erg}}
\newcommand{\sr}{\textrm{sr}}
\newcommand{\hz}{\textrm{Hz}}
\newcommand{\nhi}{N_{\textrm{HI}}}

\newcommand{\mean}{\mathbb{E}}



\begin{document}

\section*{Draine 15.4: O Star}

Consider a runaway O star (O8V spectral type) traveling through a diffuse region with $n_{H} \simeq 0.2\,\cm^{-3}$.

\subsection*{(a): Str\"omgren radius}

By equating the rates of photo-ionization and radiative recombination gives the steady state condition for the balance:
\begin{equation}
    Q_0 = \frac{4 \pi}{3} \sradius^3 \alpha_B \hdensity n_e,
\end{equation}
thus if $\hdensity = n_e = n_H$ (according to balance condition)
\begin{equation}
    \sradius = \left(
        \frac{3 Q_0}{4 \pi n_H^2 \alpha_B} 
        \right)^{1/3}
\end{equation}

Plugging the numbers:
\begin{equation}
    \begin{split}
        Q_0 &= 10^{48.44} \second^{-1} \\
        n_H &= 0.2 \, \cm^{-3} \\
        \alpha_B &\simeq 2.56 \times 10^{-13} T_4^{-0.83} \cm^3 \, \second^{-1} \\
        T_4 &= 1  
    \end{split}
\end{equation}

This gives:
\begin{equation}
    \sradius = 4 \times 10^{20} \cm.
\end{equation}

\subsection*{(b): travelling star}

Time requires traveling \Stroradius with $v_{*} = 100\, \kms$ is:
\begin{equation}
    t_{\textrm{travel}} = \sradius / v_{*}.
\end{equation}

The timescale for recombination is:
\begin{equation}
    t_{rec} = \frac{1}{\alpha_B \hdensity} 
    = \frac{\frac{4}{3} \pi \sradius^3 \hdensity}{Q_0}
\end{equation}

The comparison can be done by taking the ratio of recombination timescale and traveling timescale:
\begin{equation}
    \textrm{ratio} 
    = \frac{t_{travel}}{t_{rec}} 
    = \frac{ 
        \sradius / v_{*}
     }{
        \frac{\frac{4}{3} \pi \sradius^3 \hdensity}{Q_0}
     }
    = \frac{
        Q_0
    }{
        v_{*} {\frac{4}{3} \pi \sradius^2 \hdensity}
    }
    = 205.
\end{equation}

\subsection*{(c): implication of the comparison in item (b)}

Ionization responds on a timescale short compared to the star travelling time.
But if we have low hydrogen density, it could help to decrease the ratio, which means low density system has comparable ionization time and star travelling time. 

\section*{2: UVB background (produced by stars and SGN)}

\subsection*{(a): Photo-ionization rate}

Relevant given values:
\begin{equation}
    \begin{split}
        z &= 3\\
        \mean I_\nu(\nu_0) &= 10^{-21} \erg \cm^{-2} \second^{-1} \hz^{-1} \sr^{-1} \\
        h \nu_0 &= 13.6 \, \textrm{eV}
    \end{split}
\end{equation}

The intensity at given frequency:
\begin{equation}
    I_\nu = I_\nu(\nu_0 = 13.6 \, \textrm{eV}/ h) \nu^{-1.5}/\nu_0^{-1.5}.
\end{equation}

Based on the values given, obviously we can only compute photo-ionization rate averaged on a surface ($q_0$):
\begin{equation*}
    Q_0 = q_0 / \nhi,
\end{equation*}
with $\nhi$ in unit of $\cm^{-2}$ for the surface density.

Photo-ionization rate is the integral:
\begin{equation}
    \begin{split}
        q_0 &= \int_{\nu_0}^{\infty} \frac{I_\nu}{h \nu} d\nu
        = \frac{I_\nu(\nu_0)}{h \nu_0^{-1.5}}
        \int_{\nu_0}^{\infty} 
        \nu^{-2.5} d\nu
        = \frac{I_\nu(\nu_0)}{h \nu_0^{-1.5}}
        \times \frac{1}{1.5} \nu_0^{-1.5} 
        = \frac{I_\nu(\nu_0)}{ 1.5 \,h }\\
        &= 10^{-21} \erg \cm^{-2} \second^{-1} \hz^{-1} \sr^{-1} \frac{1}{1.5\,h}
        = 0.66 \times 10^{-21}\erg \cm^{-2} \second^{-1} \hz^{-1} \sr^{-1} h^{-1}.
    \end{split}
\end{equation}


\subsection*{(b): {\Stroradius} (or length in this case)}

Some given values:
\begin{equation}
    \begin{split}
        \hdensity &\simeq 0.1\, \cm^{-3}\\
        T_4 &= 1.
    \end{split}
\end{equation}

Do the same thing as problem one:
\begin{equation}
    Q_0 = \pi R^2 H_{S0}\, \alpha_B \hdensity n_e,
\end{equation}
where $H_{S0}$ is the height of the disk since we are physicists we only know spheres, cylinders, and boxes.

With a little bit massage, assuming $q_0$ is the photo-ionization rate from the surface of the disk:
\begin{equation}
    q_0 = H_{S0} \alpha_B \hdensity n_e
    = H_{S0} \alpha_B n_{H}^2,
\end{equation}
where the same argument applied for $\hdensity = n_e = n_{H}$.

Thus:
\begin{equation}
    H_{S0} = \frac{q_0}{\alpha_B n_H^2}
\end{equation}

By the same assumptions:
\begin{equation*}
    \begin{split}
        n_H &= 0.1 \, \cm^{-3} \\
        \alpha_B &\simeq 2.56 \times 10^{-13} T_4^{-0.83} \cm^3 \, \second^{-1} \\
        T_4 &= 1, 
    \end{split}
\end{equation*}
which gives:
\begin{equation}
    H_{S0} = 3.89 \times 10^{19} \, \cm.
\end{equation}

\subsection*{(c) maximum total column density}

Naive guess would be just the average column density in the ionized disk if we assume light is coming into the face-on disk:
\begin{equation}
    \nhi = H_{S0} \times n_H = 3.89 \times 10^{19}
    \times 0.1 \, \cm^{-2} = 3.89 \times 10^{18} \, \cm^{-2}.
\end{equation} 

\end{document}
