\documentclass[12pt,letterpaper]{article}
\usepackage{fullpage}
\usepackage[top=2cm, bottom=4.5cm, left=2.5cm, right=2.5cm]{geometry}
\usepackage{amsmath,amsthm,amsfonts,amssymb,amscd}
% \usepackage{lastpage}
\usepackage{enumerate}
\usepackage{fancyhdr}
% \usepackage{mathrsfs}
\usepackage{xcolor}
\usepackage{graphicx}
\usepackage{listings}
\usepackage{hyperref}

\usepackage{float}

% define vector
\newcommand{\q}{\underline}
\newcommand{\mt}{\mathrm}

\setlength{\parindent}{0.2in}
\setlength{\parskip}{0.1in}

% Edit these as appropriate
\newcommand\course{Phys 213}
\newcommand\hwnumber{1}                  % <-- homework number
\newcommand\NetIDa{M.-F. Ho}           % <-- NetID of person #1
% \newcommand\NetIDb{netid12038}           % <-- NetID of person #2 (Comment this line out for problem sets)

\pagestyle{fancyplain}
\headheight 35pt
\lhead{\NetIDa}
% \lhead{\NetIDa\\\NetIDb}                 % <-- Comment this line out for problem sets (make sure you are person #1)
\chead{\textbf{\Large Homework 7}}
\rhead{\course \\ \today}
\lfoot{}
\cfoot{}
\rfoot{\small\thepage}
\headsep 1.5em

\newcommand{\Data}{\mathcal{D}}
\newcommand{\xvec}{\boldsymbol{x}}
\newcommand{\Xvec}{\boldsymbol{X}}
\newcommand{\Var}{\textrm{Var}}
\newcommand{\normal}{\textrm{N}}
\newcommand{\xmean}{\langle \xvec \rangle}
\newcommand{\newx}{\tilde{x}}
\newcommand{\integer}{\mathbb{N}}
\newcommand{\thetarv}{\tilde{\theta}}
\newcommand{\phirv}{\tilde{\phi}}

\newcommand{\ml}{m_{\ell}}
\newcommand{\specterms}{^{2S+1}\mathcal{L}^p_\mathcal{J}}

\newcommand{\hi}{\textrm{H\,I}}
\newcommand{\cm}{\textrm{\,cm}}
\newcommand{\cms}{\textrm{\,cm/s}}
\newcommand{\kms}{\textrm{\,km/s}}
\newcommand{\cmcm}{\textrm{\,cm}^{-2}}
% \newcommand{\hz}{\textrm{\,s}^{-1}}

\newcommand{\civ}{\textrm{C\,IV}}
\newcommand{\mgii}{\textrm{Mg\,II}}
\newcommand{\caii}{\textrm{Ca\,II}}
\newcommand{\siii}{\textrm{Si\,II}}

\newcommand{\Isky}{I_\nu^{\textrm{sky}}}
\newcommand{\Iradio}{I_\nu^{\textrm{radio}}}
\newcommand{\indicator}{\mathbb{I}}
\newcommand{\Snuradio}{S_\nu^{\textrm{radio}}}

\newcommand{\sradius}{R_{\textrm{S0}}}
\newcommand{\hdensity}{n(\textrm{H}^{+})}
\newcommand{\second}{\textrm{s}}

\newcommand{\Stroradius}{Str\"omgren radius}
\newcommand{\erg}{\textrm{erg}}
\newcommand{\sr}{\textrm{sr}}
\newcommand{\hz}{\textrm{Hz}}
\newcommand{\nhi}{N_{\textrm{HI}}}

\newcommand{\mean}{\mathbb{E}}



\begin{document}

\section*{18.2: line ratio}

\subsection*{(a): T and n}
I think the question is asking us to draw a horizontal line on Fig 18.2, which means, for O\,III, 
\begin{itemize}
    \item $ T\sim 10^4 \, K $ if $ n = 10^6 \, \cm^{-3} $
    \item $ T\sim 1.6 * 10^4 \, K $ if $ n = 10^5 \, \cm^{-3} $
    \item $ T\sim 1.8 * 10^4 \, K $ if $ n = 10^4 \, \cm^{-3} $
    \item $ T\sim 2 * 10^4 \, K $ if $ n = 10^3 \, \cm^{-3} $
\end{itemize}

For O\,II, based on Fig 18.4, 
\begin{equation}
    n_e T_4^{-1/2} \simeq 2 * 10^2 \,\cm^{-3}.
\end{equation}

So it seems $ T\sim 2 * 10^4 \, K $ if $ n = 10^3 \, \cm^{-3} $ makes more sense.

\subsection*{(b): reddening}

The reddenning is $ A(4364.4) \textrm{\AA} - A(5008.2\textrm{\AA}) = 0.31 \, \textrm{mag} $.
The line ratio contribution from dust is $10^{0.31} = 2.04$.
The new line ratio for O\,III is:
\begin{equation}
    \frac{I([OIII]4364.4)}{I([OIII] 5008.2 )}
    = 0.003 / 2.04 = 0.0015.
\end{equation}

For Fig 18.2, we have:
\begin{itemize}
    \item $ T\sim 8 * 10^3 \, K $ if $ n = 10^6 \, \cm^{-3} $
    \item $ T\sim 1.1 * 10^4 \, K $ if $ n = 10^5 \, \cm^{-3} $
    \item $ T\sim 1.3 * 10^4 \, K $ if $ n = 10^4 \, \cm^{-3} $
    \item $ T\sim 1.4 * 10^4 \, K $ if $ n = 10^3 \, \cm^{-3} $
\end{itemize}

So we only have upper limits:
$ n_e < 10^3 \,\cm^{-3}$; and $ T > 1.4*10^4 \, K $.

\section*{Heating and Cooling}

Given:
\begin{itemize}
    \item $T = 32\,000\,K$.
\end{itemize}

\subsection*{(a): center, balance photo/cooling}

Since it is in the center, we should choose $\langle\psi\rangle$ since almost all photons will produce photoionizations.

Heating rate:
\begin{equation}
    \Gamma(H \rightarrow H^+)
    \simeq \alpha_B n_H n_e \psi k T_c
\end{equation}

Cooling rate:
\begin{equation}
    \Lambda_{rr} = \alpha_B n_e n(H^+) \langle E_{rr} \rangle
\end{equation}

Known:
\begin{equation}
    \begin{split}
        T_c &= 32\,000 \, K\\
        \langle \psi \rangle &= 1.380
    \end{split}
\end{equation}

Assume $n(H^+) \simeq n(H)$.
Equate (3) and (4):
\begin{equation*}
    \langle E_{rr} \rangle = \psi k T_c = 1.380 \times k \times 32\,000
\end{equation*}

From (27.21), we have a suspicious expressions for kinetic energy:
\begin{equation}
    \langle E_{rr} \rangle
    = (2 + \gamma) k T,
\end{equation}
where $\gamma$ depends on case A and case B and roughly to be negative unity (technically $\sim$ 0.7), so $ \langle E_{rr} \rangle \sim kT$.

Thus, 
\begin{equation*}
    T \sim 1.38 \times 32000 = 44\,160.    
\end{equation*}

\subsection*{(b): mass-weighted average temperature}
So we should take into account the cross-section at this time.
Since there would only have hydrogen, we could imagine mass is proportional to number of particles, and number of particles linearly related to cross-section.

The choice of $\psi$ will switch to $\psi_0 = 0.864$, which means (based on the same calculations in (a)) $T \sim 0.86 / 0.7 \times 32000 \sim 32000 $.

\end{document}
